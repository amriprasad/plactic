\documentclass[12pt]{amsart}
\usepackage{color}
\newcommand{\rc}[1]{\textcolor{red}{#1}}
\newcommand{\bc}[1]{\textcolor{blue}{#1}}
\newcommand{\wt}{\mathrm{wt}}
\newcommand{\ip}[2]{\langle#1,#2\rangle}
\title{A Crystal Approach to the Littlewood-Richardson Rule}
\author{Amritanshu Prasad}
\address{The Institute of Mathematical Sciences, Chennai.}
\address{Homi Bhabha National Institute, Mumbai.}
\email{amri@imsc.res.in}
\newtheorem{theorem}{Theorem}[subsection]
\newtheorem{lemma}[theorem]{Lemma}
\newtheorem{corollary}[theorem]{Corollary}
\newtheorem{proposition}[theorem]{Proposition}
\theoremstyle{definition}
\newtheorem{definition}[theorem]{Definition}
\theoremstyle{example}
\newtheorem{example}[theorem]{Example}
\newtheorem{exercise}[theorem]{Exercise}
\renewcommand{\thesubsection}{\arabic{subsection}}
\begin{document}
\section{Crystal Operators}
\subsection{Definitions}
Let $w\in A_n^*$.
Pick $i\in \{1,\dotsc,n-1\}$.
For convenience, write $j=i+1$.

The operator $e_i:A_n^*\to A_n*\cup\{0\}$ is defined via an algorithm.
At any stage of this algorithm, some letters of $w$ are \emph{frozen} and will not be touched during the rest of the algorithm.
At each stage the unfrozen part of $w$ will be denoted by $\tilde w$.
As more and more letters of $w$ are frozen, they are excluded from $\tilde w$.
\begin{itemize}
\item Freeze all letters not equal to $i$ or $j$.
\item While $\tilde w$ is not of the form $i^rj^s$ for $r\geq 0, s\geq 0$:
  \begin{itemize}
  \item Freeze all segments of $\tilde w$ of the form $ji$
  \end{itemize}
\item \bc{If $\tilde w$ does not contain $j$ return $0$.}
\item \bc{Otherwise $\tilde w=i^rj^s$ with $r\geq 0$ and $s>0$.
  Form a word $v$ by changing $\tilde w$ to $i^{r+1}j^{s-1}$ and keeping the frozen part the same.}
\item return $v$.
\end{itemize}

The operator $f_i:A_n^*\to A_n^*\cup\{0\}$ is defined in a very similar manner to $e_i$.
The only difference is that the two steps in the algorithm for $e_i$ that follow its loop (displayed in \bc{blue}) are replaced by:
\begin{itemize}
\item \bc{If $\tilde w$ does not contain $i$ return $0$.}
\item \bc{Otherwise $\tilde w=i^rj^s$ with $r> 0$ and $s\geq 0$.
  Form a word $v$ by changing $\tilde w$ to $i^{r-1}j^{s+1}$ and keeping the frozen part the same.}
\end{itemize}

\begin{example}
  Take $w=342443423122211$.
  We apply $e_1$ to $w$.
  We will show the frozen part in \rc{red}.
  The word $\tilde w$ will be what is left in black.
  \begin{itemize}
  \item Freeze all letters not equal to $1$ or $2$ to get $w=\rc{34}2\rc{4434}2\rc{3}122211$.
  \item $\tilde w=22122211$ is not of the from $1^r2^s$, so we enter the loop.
  \item Freeze segments of $\tilde w$ of the form $21$ to get $w=\rc{34}\rc{2}\rc{4434}2\rc{31}22\rc{21}1$.
  \item $\tilde w=2221$ is still not of the from $1^r2^s$, so remian inside the loop.
  \item Freeze segments of $\tilde w$ of the form $21$ to get $w=\rc{34}\rc{2}\rc{4434}2\rc{31}2\rc{2211}$.
  \item $\tilde w = 22$ is now of the form $1^r2^s$, so exit the loop.
  \item Form $v=\rc{34}\rc{2}\rc{4434}2\rc{31}1\rc{2211}$.
  \item Return $v=342443423112211$.
  \end{itemize}
  
  Now we apply $e_2$ to $w$.
  \begin{itemize}
  \item Freeze all letters not equal to $2$ or $3$ to get $w=3\rc{4}2\rc{44}3\rc{4}23\rc{1}222\rc{11}$.
  \item $\tilde w = 32323222$ is not of the form $2^r3^s$, so enter the loop.
  \item Freeze all segments of the from $32$ to get $w=\rc{342}\rc{44}\rc{342}2\rc{312}22\rc{11}$.
  \item $\tilde w = 22$ is of the from $2^r3^s$ so exit the loop.
  \item $3$ does not occur in $\tilde w$ to return $0$.
  \end{itemize}
\end{example}
Let $w\in A_n^*$ and $1\leq i<n$.
We say that \emph{$e_i$ is applicable to $w$} if $e_i(w)\neq 0$.
If $e_i(w)=0$ we say \emph{$e_i$ is inapplicable to $w$}.

Think of the function $e_i:A_n^*\times A_n^*\cup\{0\}$ as a relation: $\tilde E_i\subset A_n^*\times A_n^*\cup\{0\}$, where $(w,v)\in \tilde E_i$ if $v=e_i(w)$.
Let $E_i=\tilde E_i\cap A_n^*\times A_n^*$.
Similarly define the relation $F_i\subset A_n^*\times A_n^*$.
\begin{lemma}
  \label{lemma:e-f-inv}
  The relations $E_i$ and $F_i$ on $A_n^*$ are mutual inverses.
  In other words, if $e_i$ is applicable to $w$, then $f_ie_iw=w$.
  Similarly, if $f_i$ is applicable to $w$m then  $e_if_iw=w$.
\end{lemma}
\begin{lemma}
  For every $w\in A_n^*$ and every $1\leq i<n$, $f_i(w) = (e_{n-i+1}(w^\sharp)^\sharp)$.
\end{lemma}
\subsection{Basic Properties}
\label{sec:basic-properties}
\begin{lemma}
  \label{lemma:knuth-plactic}
  If $v=e_i(w)$, $v'=e_i(w')$, and $w\equiv w'$, then $v\equiv v'$
\end{lemma}
\begin{proof}
  Check for single Knuth move, which is routine.
\end{proof}
\begin{lemma}
  \label{lemma:tab2tab}
  If $w$ is a tableau then $e_iw$ is also a tableau.
\end{lemma}
\begin{theorem}
  If $v=e_i(w)$ then $P(v)=e_i(P(w))$.
\end{theorem}
\begin{proof}
  By Lemma~\ref{lemma:knuth-plactic}, $P(v)\equiv e_i(P(w))$.
  By Lemma~\ref{lemma:tab2tab}, $P(w)$ is a tableau.
  Since each plactic class contains a unique tableau, $P(v)=e_i(P(w))$.
\end{proof}
\begin{definition}
  [Weight of a word]
  The weight of a word $w\in A_n^*$ is the vector $\wt(w)=(\mu_1,\dotsc,\mu_n)$ where $\mu_i$ is the number of times that $i$ occurs in $w$.
  Clearly, $\wt:A_n^*\to \mathbf Z_{\geq 0}^n$ is a homomorphism of monoids.
\end{definition}
Let $\xi_i$ denote the $i$th coorinate vector in $\mathbf Z^n$.
Let $\alpha_i=xi_i-xi_{i+1}$ for $i=1,\dotsc,n-1$.
For each word $w$ and each $1\leq i<n$, let
\begin{align*}
  D(w,e_i)&=\max\{j\mid e_i \text{ applies to $w$ $j$ times}\},\\
  D(w,f_i)&=\max\{j\mid f_i \text{ applies to $w$ $j$ times}\}.
\end{align*}
Let $\ip xy$ denote the usual Euclidean inner product on $\mathbf Z^n$.
\begin{lemma}
  \begin{enumerate}
  \item For every $w\in A_n^*$ and $1\leq i<n$,
    \begin{gather*}
      \wt(e_i(w)) = \wt(w)+\alpha_i,\\
      \wt(f_i(w)) = \wt(w)-\alpha_i.
    \end{gather*}
  \item For every $w\in A_n^*$ and $1\leq i<n$,
  \begin{displaymath}
    D(w,e_i)+D(w,f_i)=\ip{\alpha_i}{\wt(w)}.
  \end{displaymath}
  \item If $|i-j|>1$, then $e_ie_j(w)=e_je_i(w)$, and
  \begin{displaymath}
    D(f_j(w),f_i)=D(w,f_i).
  \end{displaymath}
  \item
    On the other hand, $f_i$ and $f_{i+1}$ need not commute.
    We have:
    \begin{displaymath}
      D(f_{i-1}(w),f_i)=
      \begin{cases}
        D(w,f_i)+1 \text{ if $f_if_{i+1}w\neq f_{i+1}f_i(w)$},\\
        D(w,f_i) \text{ otherwise}.
      \end{cases}
    \end{displaymath}
  \end{enumerate}
\end{lemma}

\subsection{Yamanouchi Words}
\label{sec:yamanouchi-words}

\begin{definition}
  [Yamanouchi and anti-Yamanouchi words]
  A word $w\in A_n^*$ is said to be a Yamanouchi word (resp. anti-Yamanouchi word) if $e_i$ (resp. $f_i$) does not apply to $w$ for any $i=1,\dotsc,w$.
  Put another way, a Yamanouchi word is one that has $D(w,e_i)=0$ for $1\leq i<n$, and an anti-Yamanouchi word is one that has $D(w,f_i)=0$ for $1\leq i<n$.
\end{definition}
\begin{exercise}
  Show that $w$ is Yamanouchi if and only if $w^\sharp$ is anti-Yamanouchi.
\end{exercise}
In what follows, it is convenient to extend $e_i$ and $f_i$ to functions $A_n^*\cup\{0\}\to A_n^*\cup\{0\}$ by setting $e_i(0)=0$ and $f_i(0)=0$.
For a word $u=i_1\dotsb i_l\in A_{n-1}^*$, define $e_u:A_n^*\to A_n^*\cup\{0\}$ by:
\begin{displaymath}
  e_u(w)=e_{i_1}e_{i_2}\dotsb e_{i_l}(w).
\end{displaymath}
Similarly, we may define $f_u$ for each $u\in A_{n-1}^*$

We say that $e_u$ (resp. $f_u$) is applicable to $w$ if $e_uw\neq 0$ (resp. $f_u\neq 0$).
Then Lemma~\ref{lemma:e-f-inv} generalizes to:
\begin{lemma}
  If $e_u$ is applicable to $w$, then $f_{u^*}(e_u(w))=w$.
  Similarly, of $f_u$ is applicable to $w$, then $e_{u^*}(f_u(w))=w$.
\end{lemma}

\begin{theorem}
  For every $w\in A_n^*$, there exists a unique Yamanouchi word $y\in A_n^*$ such that $y=e_u(w)$ for some $u\in A_{n-1}^*$.
\end{theorem}
\begin{proof}
  The word $y$ can be obtained via the following algorithm:
  \begin{itemize}
  \item 
  \end{itemize}
\end{proof}
\subsection{Robinson's Correspondence}
\label{sec:robins-corr}

\end{document}
