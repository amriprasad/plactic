\documentclass[12pt]{amsart}
\usepackage{color}
\newcommand{\rc}[1]{\textcolor{red}{#1}}
\title{A Crystal Approach to the Littlewood-Richardson Rule}
\author{Amritanshu Prasad}
\address{The Institute of Mathematical Sciences, Chennai.}
\address{Homi Bhabha National Institute, Mumbai.}
\email{amri@imsc.res.in}
\newtheorem{theorem}{Theorem}[subsection]
\newtheorem{lemma}[theorem]{Lemma}
\newtheorem{corollary}[theorem]{Corollary}
\newtheorem{proposition}[theorem]{Proposition}
\theoremstyle{definition}
\newtheorem{definition}[theorem]{Definition}
\theoremstyle{example}
\newtheorem{example}[theorem]{Example}
\newtheorem{exercise}[theorem]{Exercise}
\renewcommand{\thesubsection}{\arabic{subsection}}
\begin{document}
\section{Crystal Operators}
\subsection{Definitions}
Let $w\in A_n^*$.
Pick $i\in \{1,\dotsc,n-1\}$.
For convenience, write $j=i+1$.
The operator $e_i:A_n^*\to A_n*\cup\{0\}$ is defined via an algorithm.
At any stage of this algorithm, some letters of $w$ are \emph{frozen} and will not be touched during the rest of the algorithm.
At each stage the unfrozen part of $w$ will be denoted by $\tilde w$.
As more and more letters of $w$ are frozen, they are excluded from $\tilde w$.
\begin{itemize}
\item Freeze all letters not equal to $i$ or $j$.
\item While $\tilde w$ is not of the form $i^rj^s$ for $r\geq 0, s\geq 0$:
  \begin{itemize}
  \item Freeze all segments of $\tilde w$ of the form $ji$
  \end{itemize}
\item If $\tilde w$ does not contain $j$m return $0$.
\item Otherwise $\tilde w=i^rj^s$ with $r\geq 0$ and $j>0$.
  Form a word $v$ by changing $\tilde w$ to $i^{r+1}j^{s-1}$ and keeping the frozen part the same.
\item return $v$.
\end{itemize}
\begin{example}
  Take $w=342443423122211$.
  We apply $e_1$ to $w$.
  We will show the frozen part in \rc{red}.
  The word $\tilde w$ will be what is left in black.
  \begin{itemize}
  \item Freeze all letters not equal to $1$ or $2$ to get $w=\rc{34}2\rc{4434}2\rc{3}122211$.
  \item $\tilde w=22122211$ is not of the from $1^r2^s$, so we enter the loop.
  \item Freeze segments of $\tilde w$ of the form $21$ to get $w=\rc{34}\rc{2}\rc{4434}2\rc{31}22\rc{21}1$.
  \item $\tilde w=2221$ is still not of the from $1^r2^s$, so remian inside the loop.
  \item Freeze segments of $\tilde w$ of the form $21$ to get $w=\rc{34}\rc{2}\rc{4434}2\rc{31}2\rc{2211}$.
  \item $\tilde w = 22$ is now of the form $1^r2^s$, so exit the loop.
  \item Form $v=\rc{34}\rc{2}\rc{4434}2\rc{31}1\rc{2211}$.
  \item Return $v=342443423112211$.
  \end{itemize}
  
  Now we apply $e_2$ to $w$.
  \begin{itemize}
  \item Freeze all letters not equal to $2$ or $3$ to get $w=3\rc{4}2\rc{44}3\rc{4}23\rc{1}222\rc{11}$.
  \item $\tilde w = 32323222$ is not of the form $2^r3^s$, so enter the loop.
  \item Freeze all segments of the from $32$ to get $w=\rc{342}\rc{44}\rc{342}2\rc{312}22\rc{11}$.
  \item $\tilde w = 22$ is of the from $2^r3^s$ so exit the loop.
  \item $3$ does not occur in $\tilde w$ to return $0$.
  \end{itemize}
\end{example}

\subsection{Basic Properties}
\label{sec:basic-properties}


\begin{lemma}
  \label{lemma:knuth-plactic}
  If $v=e_i(w)$, $v'=e_i(w')$, and $w\equiv w'$, then $v\equiv v'$
\end{lemma}
\begin{proof}
  Check for single Knuth move, which is routine.
\end{proof}
\begin{lemma}
  \label{lemma:tab2tab}
  If $w$ is a tableau then $e_iw$ is also a tableau.
\end{lemma}
\begin{theorem}
  If $v=e_i(w)$ then $P(v)=e_i(P(w)$.
\end{theorem}
\begin{proof}
  By Lemma~\ref{lemma:knuth-plactic}, $P(v)\equiv e_i(P(w))$.
  By Lemma~\ref{lemma:tab2tab}, $P(w)$ is a tableau.
  Since each plactic class contains a unique tableau, $P(v)=e_i(P(w))$.
\end{proof}
\end{document}
